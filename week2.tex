%% List of combinator problems
%% http://www.angelfire.com/tx4/cus/combinator/birds.html

%% implemented algo, used tournament selection from gp lit
%% need to implement their selection
%% tested some problems and experimented with parameter settings
%% need to test more, at different random variable seeds

\title{\textbf{Week 2 Report}\\Higher-Order Genetic Programming \\for Semantic Unifiers}

\author{Larry Diehl}
\date{\today}

\documentclass{article}

\usepackage{graphicx}
\usepackage[hscale=0.7,vscale=0.8]{geometry}

\begin{document}

\maketitle

\section{Summary}

I implemented an initial version of the higher-order GP paper by
Kuchs in Haskell. 
The current version is hardcoded to the S and K combinators, and the
algorithm parameters must be given at compile time rather than run
time. Of course, I will lift these restrictions in the future.

I also briefly looked at some combinatory logic problems from a public
dataset for automated theorem provers (TPTP), which I plan to test my
implementation against Kuchs' published results.

I was able to solve a reasonable number of problems so far, but
certain complex problems that were solved by Kuchs have not been
solved by my implementation yet. However, I have not yet done
sufficient testing on those problems (I only tried a handful of random
number seeds, and played with tweaking parameters to the algorithms).

Additionally, I found a list of combinatory logic combinators
\footnote{\texttt{http://www.angelfire.com/tx4/cus/combinator/birds.html}},
which includes combinator definitions in lambda form, SK form, and a
smaller form that reuses previously defined combinators.

The method of selection used by Kuchs was to limit selection to the
top 30 percent of programs. As suggested by GP literature
(the book ``A Field Guide Guide to Genetic Programming'')
, I instead implemented tournament selection. However, I should still
implement Kuchs' selection method to more accurately compare my
implementation to his published results.

\end{document}


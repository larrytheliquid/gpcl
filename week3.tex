\title{\textbf{Week 3 Report}\\Higher-Order Genetic Programming \\for Semantic Unifiers}

\author{Larry Diehl}
\date{\today}

\documentclass{article}

\usepackage{graphicx}
\usepackage[hscale=0.7,vscale=0.8]{geometry}

\begin{document}

\maketitle

\section{Summary}

%% 

I noticed that Kuchs had one one particular problem in his paper,
evolving the U combinator, that my implementation was not able to
solve. After much debugging, I discovered a bug in my program where
the initial population was not sorted by the fitness score. Once I
fixed this, solving U became possible.

I created a DSL for inputting unification problems from combinatory
logic, and started adding the problems from
\footnote{\texttt{http://www.angelfire.com/tx4/cus/combinator/birds.html}}
to my problem set. Then, I modified my program to print results for
multiple runs over multiple problems.

There are several difficult problems from my problem set that I cannot
yet solve. After inspecting the output of my program, I noticed that
unsolved problems follow a particular pattern. Almost all generations
contain a majority of individuals with a fitness score of 1. This
means that the algorithm can find many approximate solutions, but can
no longer tell which is better. Going forward, I have two plans to try
to solve harder combinatory logic problems:

\begin{itemize}
\item Try adding the B and C combinators, in addition to S and K, to
  the evolved solutions. The disadvantage of this approach is that the
  problem space becomes larger to explore. However, the advantage is
  that many problems can be given shorter solutions with B and C, as
  they are common specializations of S.

\item Try modifying the fitness score. I will keep the current score
  as is, and add an additional lexicographic component to fitness.
  This means that the new fitness will only be used if there is a tie
  in the current fitness (i.e. to further differentiate individuals
  with fitness 1). Many difficult solutions to equational problems
  contain contained repeated/shared variables in their solutions. I
  plan to try a an additional fitness measure based on the sum of the
  number of shared variables appearing in the solution.
\end{itemize}

\end{document}


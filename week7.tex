\title{\textbf{Week 7 Report}\\Higher-Order Genetic Programming \\for Semantic Unifiers}

\author{Larry Diehl}
\date{\today}

\documentclass{article}

\usepackage{graphicx}
\usepackage[hscale=0.7,vscale=0.8]{geometry}

\begin{document}

\maketitle

\section{Summary}

I used the new architecture (multiple fitness cases, arbitrary-term
inputs, parameterization over combinators and rewrite-rules) to
implement traditional boolean logic GP problems in this
combinator-based (Hilbert) system.

I also changed much of the architecture in preparation for turning the
project in, so I can make it runnable by command-line by someone other
than myself. Previously I had hard-coded constants for algorithm
parameters, which are now passed to the program as a problem
description argument. The programs output has also been changed to fit
within this new scheme. Additionally, I made a new problem-description
DSL so I can input problems with multiple fitness cases and
non-variable inputs, like the boolean problems mentioned above.
Previously, the random seed was a hard-coded constant that I would
manually change. Now, you can pass in a seed or let the program
generate one and tell you what it used later.

\end{document}


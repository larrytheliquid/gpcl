\title{\textbf{Week 6 Report}\\Higher-Order Genetic Programming \\for Semantic Unifiers}

\author{Larry Diehl}
\date{\today}

\documentclass{article}

\usepackage{graphicx}
\usepackage[hscale=0.7,vscale=0.8]{geometry}

\begin{document}

\maketitle

\section{Summary}

Last week I started testing GP evolving solutions to problems
involving datatypes, but they were Church-encoded datatypes (i.e. they
were represented using lambdas). This week I started to move the
project towards being able to evolve solutions using primitive
datatypes and operations over them. One way to extend combinatory
logic to have extra primitives ends up beign Illative Combinatory
Logic, which I thought I was going to use originally. However, there
are tricky issues regarding logical consistency in the illative
version, so what I ended up using was more like a Hilbert system
(regular combinators like S+K and axioms, rather than strange
combinators like $\Xi$ that can lead to inconsistency).

In previous weeks I have tested evolving combinatory logic problems
with S+K, S+K+I, S+K+B+C, and S+K+I+B+C. My project had the logic for
combinator use hardcoded, so any such change required me to edit old
code and recompile things. This week I changed my architecture so the
base combinators, and their reduction semantics, is independent
of the rest of the code (so I don't have to touch or recompile old
code when testing a new axiomatic basis).

I have also allowed problems to be supplied with multiple fitness
cases, instead of just one. Additionally, the fitness cases are now
parameterized over arbitrary terms instead of just variables.

All of this work together has allowed me to specify a combinator set
for boolean problems like the one below:

\begin{verbatim}
S x y z = x z (y z)
K x y = x
If True ct cf = ct
If False ct cf = cf
\end{verbatim}

And I can now specify problems with multiple fitness cases and
non-variable arguments like the one below for logical ``and''.

\begin{verbatim}
and True True = True
and True False = False
and False True = False
and False False = False
\end{verbatim}

Now I need to test many such problems to see how well the Genetic
Hilbert system is able to solve them.

\end{document}


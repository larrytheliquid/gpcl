\documentclass[mathserif]{beamer}
\usetheme{Warsaw}
\usepackage{proof}

\title{Higher-Order Genetic Programming \\for Semantic Unifiers}
\author{Larry Diehl}
\institute{Portland State University}
\date[Fall 2014]
{CS 541 - Final Project}

\newcommand{\n}[1]{\textrm{#1}}
\newcommand{\diff}[1]{\textcolor{red}{#1}}

\begin{document}

\frame{\titlepage}

\begin{frame}{Overview}

\begin{enumerate}
\item Backgound on Genetic Algorithms
\item Backgound on First-Order Genetic Programming
\item Backgound on Higher-Order Genetic Programming
\item Backgound on Combinatory Logic
\item Restricted Semantic Unification Problems
\item Technical Summary
\end{enumerate}

\end{frame}



\begin{frame}{Genetic Algorithm (GA)}

\begin{block}{Algorithm}
\begin{enumerate}
\item Random initialization of lists
\item Selection using fitness of lists
\item Crossover of lists
\item Mutation of lists
\item Restart until solution or max generations
\end{enumerate}
\end{block}

\begin{block}{Lists}
A list is an \textit{encoding} of a desired solution.
\end{block}

\end{frame}

\begin{frame}{Genetic Programming (GP)}

\begin{block}{Algorithm}
\begin{enumerate}
\item Random initialization of \diff{trees}
\item Selection using fitness of \diff{trees}
\item Crossover of \diff{trees}
\item Mutation of \diff{trees}
\item Restart until solution or max generations
\end{enumerate}
\end{block}

\begin{block}{Trees}
\begin{itemize}
\item Trees are \textit{concrete} program abstract syntax trees (AST's).
\item Fitness involves \textit{running} the programs.
\end{itemize}
\end{block}

\end{frame}

\begin{frame}{First-Order GP Crossover}

\begin{block}{Male}
$3 + 7$
\end{block}

\begin{block}{Female}
$n * 2$
\end{block}

\begin{block}{Child}
$n + n$
\end{block}

\end{frame}

\begin{frame}{Higher-Order $\lambda$-GP Crossover}

AST branches are applications or lambda bindings, and leaves are
variables.

\begin{block}{Male}
$\lambda x. ~ x ~ x$
\end{block}

\begin{block}{Female}
$\lambda y. ~ y$
\end{block}

\begin{block}{Child}
\diff{$\lambda x. ~ y$}
\end{block}

\end{frame}

\begin{frame}{Combinatory Logic}
\begin{block}{Primitive Combinators}
\begin{align*}
S &= \lambda f,g,x . ~ f ~ x ~ (g ~ x)\\
K &= \lambda x,y . ~ x\\
\end{align*}
\end{block}

\begin{block}{Intuitionstic Logic}
\begin{itemize}
\item
The combinators S and K are sound and complete for the implicational
fragment of intuitionstic propositional logic!
\item
Just like $\lambda$-terms, combinator expressions can represent
higher-order programs.
\end{itemize}
\end{block}

\end{frame}

\begin{frame}{Higher-Order Combinatory-GP Crossover}

AST branches are only applications, and leaves are only combinators.

\begin{block}{Male}
$(SK)S$
\end{block}

\begin{block}{Female}
$K(SS)$
\end{block}

The child below is the identity combinator $I = \lambda x . ~ x$

\begin{block}{Child}
$(SK)K$
\end{block}

\end{frame}

\begin{frame}{Combinatory Logic Problems}

\begin{block}{Problem description}
Find a combinatorial expression that is extensionally equivalent to a $\lambda$-term.
\end{block}

\begin{block}{Desired combinator as $\lambda$-term}
$T = \lambda x,y . ~ y ~ x$
\end{block}

\begin{block}{Solution as combinatorial expression}
$T = (S(K(S((SK)K))))K$
\end{block}

\end{frame}

\begin{frame}{Semantic Unification Problems}

\begin{block}{Problem description}
An equation has several universally quantified variables and a single
existentionally quantified. Find a solution to the existential that
makes the equation true after reductions. The LHS is the existential
applied to the universal variables, and the RHS is an arbitrary
expression that can mention the universal variables but not the
exitential variable.
\end{block}

\begin{block}{Desired true equation after existential substitution}
$\forall x,y . ~ \exists T . ~ T ~ x ~ y = y ~ x$
\end{block}

\begin{block}{Solution as combinatorial expression}
$T = (S(K(S((SK)K))))K$
\end{block}

GP fitness is a structural equality with the RHS, weighted by a
minimum complexity requirement.

\end{frame}


\begin{frame}{Primitives Set of Combinators}

\begin{align*}
S ~ a ~ b ~ c &= a ~ c ~ (b ~ c)\\
K ~ a ~ b &= a\\
I ~ a &= a\\
B ~ a ~ b ~ c &= a ~ (b ~ c)\\
C ~ a ~ b ~ c &= a ~ c ~ b
\end{align*}

\end{frame}


\begin{frame}{Primitives Set of Combinators}

\begin{align*}
S ~ a ~ b ~ c &= a ~ c ~ (b ~ c)\\
K ~ a ~ b &= a\\
B ~ a ~ b ~ c &= a ~ (b ~ c)\\
C ~ a ~ b ~ c &= a ~ c ~ b\\
\n{If} ~ \n{true} ~ a ~ b &= a\\
\n{If} ~ \n{false} ~ a ~ b &= b
\end{align*}

\end{frame}

\begin{frame}{Semantic Unification Problems}

\begin{block}{Problem description}
... The LHS is the existential
applied to arbitrary expressions that can mention the universal
variables, and the RHS is an arbitrary
expression that can mention the universal variables. There may be
multiple equations that are conjoined together.
\end{block}

\begin{block}{Desired true equations after existential substitution}
\begin{align*}
&\forall x,y . ~ \exists \n{and} .\\
&\n{and} ~ \n{true} ~ \n{true} = \n{true}\\
&\land \n{and} ~ \n{false} ~ y = \n{false}\\
&\land \n{and} ~ x ~ \n{false} = \n{false}
\end{align*}
\end{block}

GP treats conjoined equations as multiple fitness cases.

\end{frame}


\begin{frame}{Church-Encoded Datatypes}

Constructors of datatypes are encoded as functions representing the
cases of their eliminators.

\begin{block}{Boolean constructors}
\begin{itemize}
\item true
\item false
\end{itemize}
\end{block}

\begin{block}{Boolean eliminator}
\begin{itemize}
\item $\n{if} ~ \n{true} ~ \n{then} ~ x ~ \n{else} ~ y = x$
\item $\n{if} ~ \n{false} ~ \n{then} ~ x ~ \n{else} ~ y = y$
\end{itemize}
\end{block}

\begin{block}{Church-encoded boolean constructors}
\begin{itemize}
\item $\lambda a,b. ~ a$
\item $\lambda a,b. ~ b$
\end{itemize}
\end{block}

\begin{block}{Church-encoded boolean eliminator}
The eliminator is implicit in the constructors!
\end{block}



\end{frame}


\begin{frame}{Technical Summary}

\begin{block}{``Evolving Combinators'' by Matthias Fuchs}
\begin{itemize}
\item Problem representation is AST's of combinator expressions that
  do not mention variables.
\item Fitness is structural difference, weighted by minimum complexity.
\end{itemize}

\begin{block}{My Contributions}
\begin{itemize}
\item Single-level Tournament selection.
\item Mutation.
\item Correct-by-construction crossover.
\item Good set of primitive combinators for combinatory logic problems.
\item Partial evaluation of candidate solutions.
\item Attempt at extension to less restrictive semantic unification problems.
\end{itemize}
\end{block}



\end{block}


\end{frame}




\end{document}


\title{\textbf{Week 1 Report}\\Higher-Order Genetic Programming \\for Semantic Unification}

\author{Larry Diehl}
\date{\today}

\documentclass{article}

\usepackage{graphicx}
\usepackage[hscale=0.7,vscale=0.8]{geometry}

\begin{document}

\maketitle

\section{Summary}

I read the ``Evolving Combinators'' paper by Kuchs, which I will be
using as a basis for this project. The Kuchs paper covers how to
evolve higher-order programs with GP. To more deeply understand the
ideas, I reviewed Koza's classical first-order GP, and compared the
approaches taken by Koza and Kuchs.

The major differences between Koza and Kuchs lie in the encoding of
population individuals, and the fitness function. Both Koza and Fuchs
are covered below, focusing on encoding and fitness.

\section{First-Order Koza GP}

Koza's classic GP approaches are used to evolve first-order programs.
For example, one might evolve a program that takes 2 integers as
inputs and returns the first integer raised to the second. Another
example would be to take 2 boolean inputs and return results
that model some boolean circuit.

\subsection{Encoding}

Because the inputs and outputs are
just first-order data, the population of evolved programs consists of
a function set (branches in a tree, where the branching factor equals
the arity of the function) and a terminal set (leaves in a tree, which
may be a primitive value like an integer, or a variable for one of the
inputs to the function). 

\subsection{Fitness}

A fitness score is used to determine which members of the population
should be bred to create the next generation. For a fitness score,
a higher number is better. Sometimes an ``error'' score is used
instead of fitness, where a lower number is better (an error of 0 is
a program with maximum fitness).

In Koza's GP, it is typical to have a large number of fitness cases.
For example, you might have a complete boolean function and try to
evolve a circuit with the smallest number of boolean logic gates. In
such a case, a fitness score can be primarily based on the number of
input/output cases the program successfully matched. For functions
with infinite domains, such as evolving the exponentiation function on
integers, you can take a distribution of integers and still have a
large number of fitness cases.

If you happen to have a smaller number of fitness cases, you must have
a more clever fitness function. For example, if the output of fitness
cases are integers, you might not only want to consider a boolean
pass/fail fitness score, but consider the absolute value of the
difference between the evolved output and the desired output.

\section{Higher-Order Fuchs GP}

Fuchs evolves higher-order programs, in order to solve problems like
finding a value to subsitute into an existential variable of a formula
containing an equation with universally quantified variables.

\subsection{Encoding}

Unlike Koza's first order unification, inputs to evolved programs may
themselves be (higher-order programs), not just primitive data like
integers or booleans. A naive approach may be to use lambda's at
branches, and variables at leaves. However, then crossover would not
be very meaningful because free/captured variables might get crossed
over into a program not using them.

Instead, fuchs evolves programs using SKI (and other) combinators of
combinatory logic. Programs consisting of combinators can still take
and produce higher-order inputs and outputs. The encoding of
population used by Kuchs is particularly simple, consisting of
application nodes for branches and combinators at leaves. Variables
are not represented explicitly, and are only passed in when
calculating fitness.

\subsection{Fitness}

To calculate fitness, one must figure out if one equation equals
another after performing reductions/rewrites due to combinators. Since
you only have ``one'' fitness cases, the fitness score must be more
clever. A structural difference comparision is used, that calculates
similarity/difference based on the number of shared application nodes
containing identical sides, equal leaves, or an unequal number of
branches versus leaves.

\end{document}

